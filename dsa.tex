\chapter{Algoritmo de Assinaturas Digitais}
\label{cha:dsa}

Neste apêndice nos aprofundaremos no tema das assinaturas digitais apresentando esquemas de identificação e o algoritmo DSA.

\section{Esquemas de Identificação}
\label{sec:esqu-de-ident}

Um esquema de identificação é um protocolo em que uma parte tem como objetivo provar sua identidade para a outra.
Chamamos de {\em provador} aquele que deseja provar sua identidade e {\em verificador} aquele que deseja verificá-la.
Assumimos que o verificador possui a chave pública do provador e vamos focar em protocolos com três interações e três algoritmos $langle \mathcal{P}_1, \mathcal{P}_2, \mathcal{V} \rangle$:
\begin{enumerate}
\item O provador usa $\mathcal{P}_1$ com sua chave secreta $sk$ para gerar uma {\em mensagem inicial} $I$ e um estado $st$ e envia $I$ para o verificador.
\item O Verificador escolhe um {\em desafio} $r$ aleatoriamente de um conjunto $\Omega_{pk}$ gerado a partida da chave pública do Provador e envia $r$  de volta.
\item O Provador usa $\mathcal{P}_2$ com entradas $sk$, $st$ e $r$ para gerar uma resposta $s$ que ele envia para o Verificador.
\item Por fim, o Verificador testa se $\mathcal{V}(pk, r, s)$ computa $I$.
\end{enumerate}

% DIAGRAMA!!!

A ideia por trás do esquema de identificação é que apenas que possui a chave secreta seria capaz de $s$ a partir do desafio $r$.
A mensagem $I$ e o estado $st$ servem para garantir que um adversário não copie os mesmos passos de identificação e seja bem sucedido.
Um esquema de identificação seguro deve garantir que seja computacionalmente inviável para um adversário enganar o sistema de identificação mesmo que ele observe vários processos similares.

% agora que isso ficou no apêndice talvez faça sentido colocar a parte da transformação de Fiat-Shamir

A partir de um sistema de identificação seguro é possível gerar um sistema de assinatura digital seguro usando a {\em transformação de Fiat-Shamir} \cite{Fiat87}. 
Na hora de assinar uma mensagem $I$ e $st$ são computados e calculamos $H(I, m)$ para gerar $r$ e usamos $r$, $sk$ e $st$ para gerar $s$.
A assinatura será exatamente o par $\langle r, s \rangle$.
O algoritmo de verificação deve rodar $\mathcal{V}$ com $pk$, $r$ e $s$ como entrada para produzir $I$ e então verifica-se se $H(I, m) = r$: 

\begin{itemize}
\item $Gen(1^n) := \langle sk, pk \rangle$ e assumimos que $H: \{0,1\}^* \to \Omega_{pk}$ é uma função de hash específica
\item $Sign(sk, m) = \langle r, s \rangle$ tal que
\begin{itemize}
\item $\mathcal{P}_1(sk) := \langle I, st \rangle$,
\item $r := H(I, m)$ e
\item $s := \mathcal{P}_2(sk, st, r)$
\end{itemize}
\item $Ver(pk, m, \langle r, s \rangle) = \left\{
    \begin{array}{lcl}
      1 & \textrm{se} & H(\mathcal{V}(pk, r, s), m) = r\\
      0 & \textrm{c.c.} &\\
    \end{array}
    \right.$
\end{itemize}

Considerando que o esquema de identificação $\langle \mathcal{P}_1, \mathcal{P}_2, \mathcal{V} \rangle$ seja seguro, podemos provar que o sistema de assinatura digital acima também é seguro, mas para esta prova é necessário supor não que $H$ seja resistente a colisões, mas que ele seja um {\em oráculo aleatório}, uma suposição muito mais forte e difícil de validar empiricamente.

\section{Algoritmo de Assinatura Digital (DSA)}
\label{sec:dsa}

O esquema de assinatura RSA depende da dificuldade do problema da fatoração.
Podemos também construir sistemas que dependem da dificuldade do problema do logaritmo discreto.
Este é o caso do popular esquema DSA e do ECDSA que usa curvas elípticas.
Antes de apresentar o sistema de assinatura digital, apresentaremos um esquema de identificação em que assumimos que a chave secreta é $x$ e a chave pública é $\langle \mathbb{G}, g, y, n \rangle$ com $\mathbb{G}$ um grupo cíclico, $g$ um gerador e $y = g^x$:

\begin{enumerate}
\item O Provador sorteia $k \leftarrow \mathbb{Z}_n^\star$ e envia $I := g^k$ para o Verificador.
\item O Verificador sorteia $a, r \leftarrow \mathbb{Z}_q$ como desafios e envia de volta.
\item O Provador envia $s := [k^{-1} \cdot (\alpha + xr)\ mod\ q]$ como resposta.
\item O Verificador aceita se $s \neq 0$ e $g^{\alpha s^{-1}} \cdot y^{rs^{-1}} = I$
\end{enumerate}

A probabilidade de $s = 0$ é desprezível.
Se assumirmos que $s \neq 0$ a correção do esquema segue, pois:
\begin{displaymath}
  g^{\alpha s^{-1}} \cdot y^{rs^{-1}} = g^{\alpha s^{-1}} \cdot g^{xrs^{-1}} = g^{(\alpha + xr)\cdot s^{-1}} = g^{(\alpha + xr)\cdot k \cdot (\alpha + xr)^{-1}} = g^k = I
\end{displaymath}

É possível provar que este esquema é seguro considerando que o problema do logaritmo discreto seja seguro.

Podemos transformar esse esquema de identificação em um sistema de assinatura usando a transformação de Fiat-Shamir, mas o popular esquema DSA segue um caminho um pouco diferente.
Assumiremos a existência de duas funções $H: \{0,1\}^* \to \mathbb{Z}_n$ e $F: \mathbb{G} \to \mathbb{Z}_n$:

\begin{itemize}
\item $Gen(1^n) = \langle sk, pk \rangle$ em que:
\begin{itemize}
\item $sk := x \leftarrow \mathbb{Z}_n$ e
\item $pk := \langle \mathbb{G}, g, y, n \rangle$ com $y = g^x$
\end{itemize}
\item $Sign(sk, m) = \langle r, s \rangle$ em que:
  \begin{itemize}
  \item $r := F(g^k)$ com $k \leftarrow \mathbb{Z}_n$ e
  \item $s := [k^{-1} \cdot (H(m) + xr)\ mod\ q]$ (sorteamos outro $r$ caso $s = 0$)
  \end{itemize}
\item $Ver(pk, m, t) := \left\{
    \begin{array}{lcl}
      1 & \textrm{se} & r = F(g^{H(m) \cdot s^{-1}} y^{r \cdot s^{-1}})\\
      0 & \textrm{c.c.} &\\
    \end{array}
    \right.$
\end{itemize}

A correção desse sistema é muito similar ao esquema de identificação apresentado anteriormente.
É possível provar sua segurança assumindo a dificuldade do problema do logaritmo discreto e tratando $H$ e $F$ como oráculos aleatórios.
No caso de $F$ isso é particularmente difícil de aceitar e não se conhece uma prova de segurança melhor do que essa até hoje.