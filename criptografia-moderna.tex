\chapter{Criptografia Moderna}
\label{cha:criptografia-moderna}

No capítulo anterior apresentamos uma primeira tentativa de definir formalmente segurança de sistemas de criptografia.
A definição, proposta por Shannon, estabelece que um sistema garante o {\em sigilo perfeito} se a cifra não guarda nenhuma informação da mensagem que a produziu.
No final do capítulo, porém, vimos que esta definição não é muito útil na prática, pois obriga o universo de chaves a ser pelo menos tão grande quanto o universo das mensagens.

Apesar do fracasso desta primeira tentativa de formalizar o conceito de segurança, não abandonaremos a ideia geral.
A abordagem da {\em criptografia moderna} \cite{Goldwasser84}, que utilizaremos nesta apostila, segue três princípios básicos:
\begin{enumerate}
\item {\em definições formais:} As noções de segurança utilizadas serão apresentadas de maneira formal por meio de definições. 
As definições prévias nos ajudam a comparar sistemas e avaliar sua segurança a partir de critérios estabelecidos previamente.
\item {\em suposições explícitas:} Na maioria dos casos seremos forçados a fazer suposições sobre os sistemas de criptografia que não seremos capazes de demonstrar.
Ainda assim, é imprescindível explicitar essas suposições de maneira clara e formal.
Nossa incapassidade de provar tais suposições não nos impede de validá-las empiricamente.
Grande parte do trabalho envolvido na criptografia moderna consiste em validar testar as suposições sobre um sistema e buscar suposições mais simples e básicas.
\item {\em demonstrações formais:} Quando somos capazes de formalizar nossas suposições e a definição de segurança desejada, eventualmente podemos demonstrar que um sistema que satisfaz as suposições garante determinada noção de segurança.
Esse tipo de demonstração reduz o problema da segurança às suposições do sistema que devem ser mais simples e mais fáceis de validar empiricamente.
Esta redução permite que se substitua um sistema cuja suposição foi falseada antes que ele seja quebrado.
\end{enumerate}

As definições de segurança em geral possuem dois componentes: uma garantia de segurança -- o que pode ser considerado um ataque bem sucedido -- e um modelo de ameaças.
Por exemplo, na definição de {\em sigilo perfeito} a garantia de segurança é que nenhuma informação sobre a mensagem esteja contida na cifra -- ou como formulamos, a probabilidade de ocorrência da cifra deve ser independente da probabilidade de ocorrência da mensagem -- e o modelo de ameaça assume que o adversário tem acesso apenas ao texto cifrado e nada mais. 
Os modelos de ameaças que estudaremos no livros incluem:
\begin{itemize}
\item {\em ataque ciphertext-only:} Este é o modelo assumido na definição de sigilo perfeito.
Nele assumimos que o adversário tem acesso apenas a um texto cifrado de tamanho arbitrário.
\item {\em ataque chosen-plaintext:} Neste modelo, além de assumir que o adversário tem acesso à cifra, assumimos que ele é capaz de escolher uma quantidade de mensagens e verificar como elas seriam cifradas pelo sistema.
\item {\em ataque chosen-ciphertext:} Neste modelo assumimos que o adversário é também capaz de escolher certas cifras e verificar como elas seriam decifras pelo sistema.
\end{itemize}

Essas definições são progressivamente mais fortes, ou seja, assumem progressivamente maior capacidade de ataque do adversário.
Note que nossos modelos de ataque não fazem qualquer suposição sobre a estratégia utilizada pelo adversário.
Em geral, a definição mais adequada depende das necessidades do problema em mãos -- eventualmente pode ser desejável um sistema mais fraco e mais eficiente.

Nossa primeira tentativa de definir segurança esbarrou na limitação inconveniente expressa pelo teorema de Shannon.
Afirmamos no capítulo anterior que o sigilo perfeito pode ser definido por meio de um jogo.
O sigilo perfeito é garantido se nenhum adversário for capaz de distinguir qual mensagem foi encriptada pelo sistema.
Para contornar as limitações expressas pelo teorema de Shannon, enfraqueceremos este tipo de definição de duas formas:
\begin{itemize}
\item O adversário deve ser {\em eficiente} e usar sua estratégia em um tempo razoável.
  Ou seja, eventualmente um adversário pode derrotar o sistema desde que seja dado a ele tempo suficiente.
  A existência de um adversário como este não violará nossa definição de segurança, pois não traz uma real ameaça na prática.
\item O adversário pode eventualmente derrotar o sistema, mas com uma {\em probabilidade muito pequena}.
Colocado de outra maneira, a cifra pode guardar alguma informação sobre a mensagem desde que seja muito pouca. 
\end{itemize}

\section{Abordagem Assintótica}
\label{sec:abord-assint}

Em termos práticos o que buscamos uma noção de segurança em que para um adversário ter sucesso ele precisaria rodar seu algoritmo em uma máquina excelente por um intervalo bem grande de tempo.
Alternativamente esperamos que eventualmente o adversário derrote o sistema em pouco tempo, mas com uma probabilidade muito baixa.
O problema com essa abordagem é como definir o que seria uma ``máquina excelente'', um ``intervalo bem grande de tempo'' e uma ``probabilidade muito baixa''.
Essas questões são particularmente complicados em um contexto em que a capacidade computacional evolui de maneira acelerada.

Ao invés de estabelecer valores fixos para definir esses limites, partiremos de uma {\em abordagem assintótica}.
Suporemos que o algoritmo de geração de chaves $Gen$ recebe um {\em parâmetro de segurança} $n$ e estabeleceremos nossos limites em função deste parâmetro -- tipicamente denotaremos esse valor usando a notação unária $1^n$, pois a tempo de execução costuma ser calculado como uma função do tamanho da entrada.
Para os efeitos desta apostila podemos assumir que o parâmetro está relacionado ao tamanho da chave a ser gerada e que é de conhecimento público.
O parâmetro de segurânça permite que as partes ajustem seu sistema para o nível desejado de segurança -- aumentar seu tamanho costuma refletir em um aumento no tempo de processamento do sistema e em um aumento no tamanho da chave, portanto, quem projeta o sistema tem um incentivo para querer minimizá-lo as custas de diminuir sua segurança.
A capacidade de ajustar o parâmetro de segurânça possui grandes vantagens práticas, pois permite se defender de adversários com poder computacional mais forte com o passar do tempo. 

Vamos estabelecer que o adversário buscará quebrar a cifra usando um algorítmo randomizado polinomial em $n$ e que ele pode vencer com uma probabilidade desprezível em $n$.
A suposição de que o adversário usa um algoritmos randomizado é forte, significa que ele é capaz de acessar uma quantidade arbitrária de bits aleatórios.
Isso certamente não é possível na prática, mas serve bem aos propósitos de uma análise de pior caso.
Já a probabilidade ser {\em desprezível} significa que ela cresce assintoticamente mais devagar do que o inverso de qualquer polinômio.
Formalmente, $\varepsilon: \mathbb{N} \to \mathbb{N}$ é {\em desprezível} se para todo polinômio $p$ existe um número positivo $N$ tal que para todo $n > N$ temos que $\varepsilon(n) < \frac{1}{p(n)}$.

Estamos agora em condições de reescrever a definição de segurança formalmente incorporando os enfraquecimentos descritos neste capítulo.

\begin{definition}
  Considere o jogo apresentado no Capítulo \ref{cha:sigilo-perfeito}, um sistema $\Pi = \langle Gen, E, D \rangle$ é seguro contra ataques {\em ciphertext-only} se para todo adversário polinomial $\mathcal{A}$ existe uma função desprezível $\varepsilon$ tal que para todo $n$:
\begin{displaymath}
  Pr[PrivK^{eav}_{\Pi, \mathcal{A}}(n) = 1] \leq \frac{1}{2} + \varepsilon(n)
\end{displaymath}
\end{definition}

Lembrando que agora o algoritmo $Gen$ recebe como o parâmetro de segurança $n$ em notação unária para gerar a chave de tamanho apropriado.
Em palavras, a definição estabelece que um sistema é seguro contra ataques {\em ciphertext-only} se nenhum algoritmo eficiente é capaz de derrotá-lo com probabilidade consideravelmente maior do que $\frac{1}{2}$.
Resta mostrar um sistema que satisfaça essa definição.


\section{Exercícios}
\label{sec:exercicios}


\begin{exercicio}
  Sejam $\varepsilon_1$ e $\varepsilon_2$ duas funções desprezíveis. 
  Mostre que $\varepsilon_1 + \varepsilon_2$ é desprezível.
\end{exercicio}

\begin{exercicio}
  Seja $\varepsilon$ uma função desprezível e $p$ um polinômio. 
  Mostre que $p\varepsilon$ é desprezível.
\end{exercicio}

\begin{exercicio}
  Quais as vantagens da abordagem assintótica na definição da garantia de segurança?
\end{exercicio}

\begin{exercicio}
  Por que representamos o parâmetro de segurança em notação unária?
\end{exercicio}

\begin{exercicio}
  Considere um sistema $\Pi$ seguro contra ataques {\em ciphertext only} cujo parâmetro de segurança tem $128$ bits ($n = 128$) e um adversário polinomial que derrota o sistema com probabilidade $\frac{1}{2} + \frac{1}{2^{n/4}}$.
Com que probabilidade esse adversário derrotaria o sistema se dobrassemos $n$?
\end{exercicio}

\begin{exercicio}
  Considere um sistema $\Pi$ seguro contra ataques {\em ciphertext only} cujo parâmetro de segurança tem $128$ bits ($n = 128$) e um adversário que roda um algoritmo que derrota o sistema com probabilidade $1$ em $2^{n/8}$ passos.
  Quantos passos seriam necessários para esse algoritmo derrotar um sistema se dobrássemos $n$? E se quadrupicássemos $n$?
\end{exercicio}